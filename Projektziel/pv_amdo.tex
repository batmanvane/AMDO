\documentclass[12pt]{amsart}
\usepackage[latin1]{inputenc}
\usepackage{amssymb}
\date{\today}

\begin{document}
\section{Optimierungsproblem f�r die Auslegung einer PV Freifl�chenanlage}

{\bf Gegeben}
\begin{itemize}
\item Lastgang des Energieverbrauchs \( P_{\text{Last}}(t) \) �ber einen bestimmten Zeitraum.
\item Kosten f�r die Errichtung der PV-Anlage, Investitionskosten \( C_{\text{Invest}} \).
\item Betriebs- und Wartungskosten der PV-Anlage �ber einen bestimmten Zeitraum \( C_{\text{O\&M}} \).
\end{itemize}

{\bf Entwurfsvariablen}

\begin{itemize}
\item Ausrichtung der PV-Module \( \theta \) (in Grad, z.B., bezogen auf geografische Nordrichtung).
\item Neigung der PV-Module \( \phi \) (in Grad, z.B., Winkel zur horizontalen Ebene).
\item Fl�chennutzengrad \( \eta_{\text{Fl�che}} \) (Verh�ltnis der tats�chlich genutzten Fl�che zur Gesamtfl�che).
\end{itemize}

{\bf Zielkriterien}

\begin{enumerate}
\item Maximierung der Autarkiegrad \( AG \):
   \[ \max AG = \frac{\int_{t_1}^{t_2} P_{\text{PV}}(t) \, dt}{\int_{t_1}^{t_2} P_{\text{Last}}(t) \, dt} \]
   Dabei ist \( P_{\text{PV}}(t) \) die von der PV-Anlage erzeugte Leistung zum Zeitpunkt \( t \) und \( P_{\text{Last}}(t) \) entsprechend angeforderte Last.

\item Maximierung des Energieertrags \( E \):
   \[ \max E = \int_{t_1}^{t_2} P_{\text{PV}}(t) \, dt \]

\item Maximierung der Wirtschaftlichkeit \( W \) (oder halt Eigenverbrauch \( EV \))):
   \[ \max W \text{bzw.} EV\]
\end{enumerate}

Unter den Nebenbedingungen:

\begin{enumerate}
\item Leistungsbilanz:
   \[ P_{\text{PV}}(t) = \eta_{\text{Fl�che}} \cdot P_{\text{Sonnenstrahlung}}(t, \theta, \phi) \]
   wobei \( P_{\text{Sonnenstrahlung}} \) die auf die PV-Module treffende Sonnenstrahlung ist.

\item  Lastgang:
 \( P_{\text{Last}}(t) \)  gegebener Lastgang.

\item  Ausrichtungsbeschr�nkung:
   \[ 0 \leq \theta \leq 360^\circ \]

\item  Neigungsbeschr�nkung:
   \[ 0 \leq \phi \leq 90^\circ \]

\item Fl�chennutzengradbeschr�nkung:
   \[ 0 \leq \eta_{\text{Fl�che}} \leq 1 \]

\item Modell Abschattungs(winkel) : \[  \eta_{\text{Fl�che}} \]

\end{enumerate}
Die Integration �ber die Zeit repr�sentiert den Zeitraum \( [t_1, t_2] \), in dem die Auslegung gelten soll.\end{document}
